\documentclass{article}
\usepackage{graphicx} % Required for inserting images
\usepackage{geometry}[margin=1in]
\usepackage[super]{nth}

\title{Women in Maritime Leadership Conference Reflection\\MGT 310 Extra Credit}
\author{Joey Simone}
\date{\today}

\begin{document}

\maketitle
\subsection*{Sessions Attended}
On March \nth{15}, I attended these sessions: \begin{itemize}
    \item Madeleine Wolczko's Opening Keynote
    \item MSC Showcase
    \item CGIS Tabletop
    \item Team Building Panel
    \item Student Panel: The Next Generation
    \item Stay Afloat at Sea or Ashore
\end{itemize}
On March \nth{16} I attended \begin{itemize}
    \item USCG Showcase
    \item DEI Best Practices
    \item Captain Bannister's Closing Keynote
\end{itemize}
I also was there for the Macy's Dress for Success raffle but that doesn't really count.
\section*{Experiences and Perspectives}
The presenter with whom I identified the most and whose keynote resonated with me was the Opening Keynote, Madeleine. A part of me wants to be just like her, always trying to advance my career, overstaying hitches to get more hands on experience and getting to spend more time with the crew. When she described how she got used to being catcalled and harassed as a young person and a young adult, I didn't share any connection to this, I have never and will never get catcalled or assaulted, but I do empathize with her feelings of fear to walk the streets. I am a pretty confident person, but my being non-binary often confuses people, and I worry about my safety in public places. If this is going to be the day that some guy will hear my voice not match my face and decide that today is when he's gonna pick a fight, or if when I get ID'd at a bar or the airport that I'll be turned away for not being clearly the person displayed.\\ I felt such relief to hear that when you get your license, many of those troubles disappear aboard ship. McNie, in her panel, had a different experience in the 80s, but the women who sailed in the 2010s and 2020s did report that they can leave their gender and the problems of feminism on land, and at sea be just a member of the crew, and that so long as one is competent in their skills, which I am, they can just be "the mate". This option is so attractive to me that I often think about going to sea for a thousand day hitch, or even a fantasy of going to sea never to return. To run away from the problems of the landsman's world from May \nth{5} 2024 until the day I die. I don't have to worry about what my flesh and blood family will think when I come out to them if the crew of the M/V Flying Dutchman will be my real family.

She has experienced a traumatic flip side of that coin. She was on a ship in dry dock in Shanghai during the lockdown in spring of 2022, and her total time aboard ship was 217 days. Her company wouldn't take her home, and her creative outlet of making the documentary series "Restricted to Ship" (available on YouTube) was her only way of coping. She entirely lost control of her situation, when usually the officer of the watch has some modicum of power, and it broke her emotionally for months after her hitch was over.

This has changed a part of my overall goal, in perhaps a "be careful what you wish for, because you just might get it" sense. I have no reason to believe I'm any stronger than she was, and in fact I know that my own will is pitifully weak. I am sure that I too would crumble under such an oppressive quarantine confined to a ship in a foreign country for months. I can't make the sea my first and last goal. I can't tie myself to the mast in the hopes that I can forever avoid the sirens and the rocks that await me ashore.
\section*{Advancing Leadership Skills}
Much of the content of the conference was the presenters giving their tips on how diverse people can be leaders and how leaders can improve diversity in a virtuous cycle of making people comfortable and confident, getting those people in positions of power, and expanding the base of your community. These overlap broadly, and I will now give a collection of the tips and mantras I collected during the sessions I attended on Saturday. \begin{itemize}
    \item You have to talk to people to build trust, you have to talk to people to make progress. Find the level of speaking up people are willing to do and make them feel safe and empowered to do that talking.
    \item Finding and interacting with community and expanding that community is the only way to change organizational culture.
    \item You can't heal what you don't reveal.
    \item if there's not a seat for you at the table \textbf{build it.}
    \item Persistence is the most valuable quality of humanity. Push through unfairness.
    \item Bets McNie had 3 tips to give:
    \begin{enumerate}
     \item Professional: Be a good shipmate, prove your salt, ask for extra work, build respect.
     \item In your community: clarify goals, ask for help, seek resources.
     \item Personal: know yourself, be introspective and solidify your core values and goals, because you can't help other people until you've helped yourself.
    \end{enumerate}
        \item Lower your defenses, people are less productive when they are defensive around each other.
        \item Show subordinates in the disadvantaged groups a path to advancement.
        \item Learn to live with less in the lean times.
        \item Better does not mean easier. Speaking up will almost never make it easier, but it will make it better.
        \item Failing, much more than getting it right the first time, helps you learn, and helps you recover from mistakes more quickly, recovery is a skill that can only be sharpened by failing at something else.
        \item Ask questions, never do something you're unsure of how to do it.
        \item Find the secret selfish pleasure that only you have, it will motivate you when things are tough.
        \item You have to try to fly under the radar and do your job, even though you are a Queen of Hearts in a deck of spades.
        \item When men talk to you, ignore the method of communication, focus on the message. It doesn't matter if it's whispered, yelled, emailed, smoke signal, or Morse code. The context of the message does not matter anywhere near as much as the actual words being said, because if they're telling you it means they still care. When they stop talking to you it means they no longer care about you.
        \item and lastly Bannister's Three Rules: \begin{enumerate}
            \item Communicate, and Don't Lie.
            \item Do your job, or someone else will have to do your job in addition to theirs.
            \item Respect everyone. You are all trained professionals.
        \end{enumerate} If you cannot do all three of those, you cannot be safe.
  
\end{itemize}
  Obviously, each and every one of those is vital to everyday operations and in VUCA times, but if any, I will pick Bannister's "Better does not mean easier." As an officer, it will be my vital duty to speak up, even and especially if it means that something will get harder for me. Maybe I'll be the person put in charge of fixing it, or maybe the old guard set in their ways will be mad at me and give me sh*t duty for daring to go against the status quo, but it would go against my honor to keep quiet.
\end{document}
