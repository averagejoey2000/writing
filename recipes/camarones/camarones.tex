\documentclass[letterpaper]{recipe}
\usepackage{spanish]{babel}
\begin{document}
\recipe{Camarones a la Diabla}
\ingred 8 chiles guajillos, enjuagados, sin tallos ni semillas;
 3 chiles de árbol, enjuagados y sin tallos;
      3 tomates roma, picados;
      2 dientes de ajo;
      1/2 cebolla blanca mediana, picada en trozos grandes;
      1 cucharadita de sal kosher;
      4 cucharadas de aceite de oliva;
      1,5 libras de camarones crudos grandes, pelados, desvenados y con la cola;
      sal y pimienta negra, al gusto}

      En un tazón o cacerola mediana, agrega los chiles guajillo y de árbol secos. Agrega agua muy caliente o hirviendo hasta que los chiles estén completamente sumergidos. Cubra con una tapa o un plato grande y déjelo reposar durante 15 minutos, hasta que los chiles se ablanden.

     Con una espumadera, transfiera los chiles ablandados a una licuadora grande. Agrega los tomates, el ajo, la cebolla y la sal. Haga puré hasta que esté completamente suave. Pruebe y sazone con más sal, si es necesario. Si la salsa está demasiado picante, agregue más tomates.

     Calienta una sartén o sartén grande a fuego medio-alto. Agrega el aceite de oliva y los camarones. Cocine los camarones durante 1 minuto por lado o hasta que estén de color rosa claro.

     Agregue la salsa de chile rojo a la sartén o sartén y mezcle para cubrir los camarones. Baje el fuego a medio y cocine de 3 a 5 minutos, hasta que la salsa esté burbujeante y caliente.

     Retira la sartén o sartén del fuego y sírvelo solo como aperitivo o con Auténtico Arroz Mexicano como comida.
     \end{document}
