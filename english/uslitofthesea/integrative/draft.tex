% Options for packages loaded elsewhere
\PassOptionsToPackage{unicode}{hyperref}
\PassOptionsToPackage{hyphens}{url}
%
\documentclass[
]{article}
\usepackage{lmodern}
\usepackage{amssymb,amsmath}
\usepackage{ifxetex,ifluatex}
\ifnum 0\ifxetex 1\fi\ifluatex 1\fi=0 % if pdftex
  \usepackage[T1]{fontenc}
  \usepackage[utf8]{inputenc}
  \usepackage{textcomp} % provide euro and other symbols
\else % if luatex or xetex
  \usepackage{unicode-math}
  \defaultfontfeatures{Scale=MatchLowercase}
  \defaultfontfeatures[\rmfamily]{Ligatures=TeX,Scale=1}
\fi
% Use upquote if available, for straight quotes in verbatim environments
\IfFileExists{upquote.sty}{\usepackage{upquote}}{}
\IfFileExists{microtype.sty}{% use microtype if available
  \usepackage[]{microtype}
  \UseMicrotypeSet[protrusion]{basicmath} % disable protrusion for tt fonts
}{}
\makeatletter
\@ifundefined{KOMAClassName}{% if non-KOMA class
  \IfFileExists{parskip.sty}{%
    \usepackage{parskip}
  }{% else
    \setlength{\parindent}{0pt}
    \setlength{\parskip}{6pt plus 2pt minus 1pt}}
}{% if KOMA class
  \KOMAoptions{parskip=half}}
\makeatother
\usepackage{xcolor}
\IfFileExists{xurl.sty}{\usepackage{xurl}}{} % add URL line breaks if available
\IfFileExists{bookmark.sty}{\usepackage{bookmark}}{\usepackage{hyperref}}
\hypersetup{
  hidelinks,
  pdfcreator={LaTeX via pandoc}}
\urlstyle{same} % disable monospaced font for URLs
\setlength{\emergencystretch}{3em} % prevent overfull lines
\providecommand{\tightlist}{%
  \setlength{\itemsep}{0pt}\setlength{\parskip}{0pt}}
\setcounter{secnumdepth}{-\maxdimen} % remove section numbering

\author{}
\date{}

\begin{document}

Synthesis of Themes: Agency in Art

Across eras, perspective as a tool for the author is their main tool to
force the reader into certain patterns of thought, and to identify with
certain characters. Moby Dick, Turnabout, and Metal Gear Solid 2 each in
different ways use the aspects of their medium to convey a story about
perspective, and shape a story about how this perspective effects the
characters and the audience. The ways in which they use perspective are
characteristic of their artistic periods, but all convey a broader
message about choice and information.

Moby Dick is a product of the Romanticism movement in art. From the
first words, the reader is made aware that this is an inner experience,
and that Ishmael is telling you a story about himself. He will paint you
beautiful pictures of what happened to him and his friends, but the
story takes place, in most chapters, inside of his head. The subject
matter of the story is the whale, but the essence of the book is to
relate to the reader. Ishmael is an unusual blend of character and
narrator, we have only him to trust about the events of the book, but he
also has access to information it would be entirely unreasonable and
unrealistic for him to have, for the purpose of dramatic content. By
showing private conversations between Starbuck and Ahab, or the internal
monologues of the same men, Ishmael provides context, and in a way
censors alternative interpretations and explanations of what happened
aboard the Pequod on her final voyage. These make way for his delivery
of various sermons on the importance of whaling to modern society,
speculations on the characteristics of cetaceans, and to judge the
reader to their face. ``Whaling not respectable?'' This represents our
pre-modern representation of how to tie agency into the medium. We have
no choice but to sit and listen to him talk about how ``Whales are fish
because 1. They Swim in the Water'' because Ishmael controls the speed
at which he tells us the main story of the book, providing we don't skip
to the end or read the CliffsNotes. An interesting case exists in ``The
Town Ho Story'', as a story he tells us about a time he told a story to
people we don't know, and he learned that story from someone else who
didn't want to tell him about it, and it becomes relevant to the action
of the story as an -independent- report of the power of Moby Dick. It
forces us out of our immersion into the events of the story and cements
our perspective as readers of Ishmael's book, all to distract us from
the perspective that we are the readers of Melville's book. Chapters
like these and The Affidavit serve as impassioned pleas by Ishmael for
us to take as Gospel Truth a story from the fiction shelf, and in return
he'll tell you about how the carpenter complained about making a coffin
for his boyfriend into a lifebuoy.

At a number of points, Ishmael ascribes the actions of various actors in
the play as being subservient to fate, including himself. He was fated
to go whaling, he was fated to choose the Pequod, Queequeg's tattoos
foretold the confrontation with the whale, fate stayed the hand of
Starbuck when he held the musket, and fate compelled Ahab to tangle with
Moby Dick for the last time, even when in the symphony he saw that ever
so human reflection of pain and hope in Starbuck's eye. The thread of
Fedallah's prophecy is a specter which looms over Ahab, he only tells
any of his crew about it once it becomes too late for them to be saved,
and he of course fatally misinterprets it. Furthermore, that this comes
in a book, and all the events are long passed. There is no suspense for
Ishmael, and he has on a number of pages mentioned the deaths of certain
characters before they actually happened. Search for the epitaph of
Bulkington or the reference to Queequeg's last dive. This comes in the
face of the romantic ideas of hero worship and self-determination. No
matter how much can be ascribed to the strength of Ahab and his
legendary status, he was no match for the whale, and it destroyed
everybody. As discussed in class, a complete reputation of the
one-great-man, and of his agency.

Turnabout is, as the name implies, a straight shot and a sharp turn,
repeatedly. We are introduced to each of the characters in a way
matching the stereotype we might have for each man, the grizzled
American bomber pilot, the cocksure but prim lieutenant, the drunk
effeminate English navy twink. Faulkner allows us to let sit those
pre-conceptions in our minds in the time leading up to the air mission,
only giving slight clues to the cleverness of Mr.~Hope. In the modern
style, the descriptions of the actions and set pieces are tight and
visual, with a keen sense of location, and little embellishment. A story
about picking up a tiny drunk boy takes a sharp turn when they take him
on an air raid, and shows his competency with an airframe mounted
machine gun, and some fine orienteering skills within seconds of
entering the plane's gunner position in the nose. Perspective is called
attention to in the story when our hero the captain disregards what Hope
said about the bomb which failed to release properly, despite him having
the vantage point to see the bomb itself hanging off the wing. The
captain's self-image as a competent and daring officer is challenged
when he enters Hope's element and embarks on the voyage in the bomb
torpedo boat on what to all the world seems like a suicide mission. Less
colorful language is used to illustrate that he is out of his depth, but
to have him fumble to unjam the torpedo and throw up trying. In a
previous class session, I remarked that every named character in
Turnabout is a soldier, and in a time when the draft was active, WWI,
desertion is punishable by death. In this way, none of the characters
have any true choice in what they are permitted to do, they are forced
by their country to fight as soldier slaves. Their agency is taken away
by their positions, and their degrees of freedom are shown by their
uniforms. We are introduced to the American captain by showing his
uniform as being ever so slightly out of regulations, and he is the one
who at the end of the short story is shown having the treasonous thought
to destroy all the world's generals, presidents, kings, and diplomats
who have forced this life of death upon him, who sent Hope on a long
range torpedo boat patrol from which he never returned. Modernism itself
comes during the time of confusion coming from the fall of Empires
surrounding the first and second world wars, and artists became more
experimental in their styles, as shown with Faulkner's fast and loose
dialogue and pacing. This is our Modern pillar, no longer forcing the
inner perspective of feelings of the author, but of the characters, and
allowing the reader to empathize or not with any of these warfighters.

The hardest part of this essay to write is the postmodern entry, Metal
Gear Solid 2, often touted as the first or most postmodern videogame. As
an interactive medium, the concept of canon is malleable, there are no
concrete answers to ``how many times was the main character spotted? How
many guards did they kill?'' and other comparable questions as to how
the player behaved in the simulation. As a sequel to an acclaimed action
game, players already have an idea of what to expect, and have their own
goals in mind for their experience of the game. Kojima, the auteur of
MGS2, revels in taking this away from the player. In the previous game,
the player, the audience, plays as a badass secret agent named Solid
Snake, who infiltrates a nuclear compound in Alaska which has been taken
over by terrorists. Most people who purchase the sequel would again
expect to be permitted to indulge in the fantasy of being that same
fantastical action hero, but MGS2 rips away not only the opportunity to
repeat their experience, but also the satisfaction they had of doing it
the first time. In the first act, the player does get to play as Snake,
but in the second act, they are forced to play as Raiden, a man who was
trained on simulations of the first game, and who himself admires snake.
This change in perspective holds up a mirror to the player. ``You are
not a badass secret agent. You played him in a videogame, this guy
played him in a videogame, you are this guy. You are not Snake, you are
Raiden'' and this sharing is meant to be incredibly uncomfortable for
both the player and for Raiden. The player is told to put their name on
Raiden's dog tags at a computer, to carve in iron that they are one.
This is the opposite of a ``Call me Ishmael'' moment. Instead of Ishmael
telling us who he is and making us feel his feelings and read his diary,
we are forced to live with Raiden's thoughts, and he is our puppet.

In the same chapter of the second act, the player is required to use the
first person mode in the game in a way that was not done in the previous
game, putting them into Raiden's headspace in a literal way in addition
to a figurative way, This is the first 20 chapters of Moby Dick.
Everyone who purchased a book called ``The Whale'' expecting an action
adventure got to sit through a monologuing unskilled sailor talking
about how handsome his husband is for a fifth of the book before we even
see Ahab, the badass onto whom John Doe projects himself. After the
halfway point through the game, the character of Solid Snake from the
previous game is fighting alongside Raiden, driving home the fact that
"you are not him. There he is, next to you, you are not James Bond, you
are not John Rambo, you are not Solid Snake. That's him and you're not
him" the game all but screams at you.

In a series of monologues delivered by government agents and a rogue AI,
it is revealed to the player via Raiden that the shadow government
controls messaging in mass media and the internet by pruning what it
qualifies as misinformation, creating signal noise (``creating context
not destroying content'' by the AI's own words), and controlling
behavior on the society scale through memes. They also control behavior
on the micro scale by creating certain pressures in an individual
person's scenario. The player's character, Raiden, was given a mission
in this game similar to the events of the first game, and they (the
shadow government) hoped that the similarities between the missions
would shape Raiden to be a supersoldier like snake, but one that they
controlled implicitly for their own ends. They control the game in a
meta sense by exploiting a Metal Gear Solid fan's inclination to try to
be Solid Snake, so the pressure also effects the person holding the
controller. About at this time, the game pulls some uniquely Kojima
tricks like creating a fake game over screen, pretending to switch the
inputs, taking control of the character away from the player unless they
change what port the controller is plugged into, reading the player's
memory card, or giving Raiden a message over the in game radio telling
him to turn off the console.\footnote{Watch these
  \url{https://www.youtube.com/watch?v=eKl6WjfDqYA}
  https://www.youtube.com/watch?v=z86oAKXUKxo} It is revealed that the
last chapter of the game has been taking place in the belly of a
mechanical whale, or what is called the sigmoid colon of Arsenal Gear, a
submersible base housing autonomous weapons of mass destruction
controlled by the shadow government robots. Raiden crashes the
whale/base/ship into New York (where Ishmael is from). In a subversion
of Captain Bogart's treasonous secret thought of blowing up the kings,
presidents, and generals, the AI coerces Raiden into killing the
President of the United States. The game has a post credits sequence,
where Raiden decides to be his own man, and throws away the dog tags
bearing the player's name, severing the connection between them, and
ending the game.

Each work of art, in their unique mediums; the first great American
novel, a short story that wouldn't be out of place as a Tarantino flick,
and a videogame which is designed from the ground up to invade the
audience's mind in the most disturbing way possible; use their formats
to convey a story about agency, about choice, about heroes and hero
worship in the face of fate, government, and unseen forces which control
your very mind and body.\footnote{If you have 6 hours, watch this
  https://www.youtube.com/watch?v=I5AunfmI8bs}

\end{document}
