% Options for packages loaded elsewhere
\PassOptionsToPackage{unicode}{hyperref}
\PassOptionsToPackage{hyphens}{url}
%
\documentclass[
]{article}
\usepackage{lmodern}
\usepackage{amssymb,amsmath}
\usepackage{ifxetex,ifluatex}
\ifnum 0\ifxetex 1\fi\ifluatex 1\fi=0 % if pdftex
  \usepackage[T1]{fontenc}
  \usepackage[utf8]{inputenc}
  \usepackage{textcomp} % provide euro and other symbols
\else % if luatex or xetex
  \usepackage{unicode-math}
  \defaultfontfeatures{Scale=MatchLowercase}
  \defaultfontfeatures[\rmfamily]{Ligatures=TeX,Scale=1}
\fi
% Use upquote if available, for straight quotes in verbatim environments
\IfFileExists{upquote.sty}{\usepackage{upquote}}{}
\IfFileExists{microtype.sty}{% use microtype if available
  \usepackage[]{microtype}
  \UseMicrotypeSet[protrusion]{basicmath} % disable protrusion for tt fonts
}{}
\makeatletter
\@ifundefined{KOMAClassName}{% if non-KOMA class
  \IfFileExists{parskip.sty}{%
    \usepackage{parskip}
  }{% else
    \setlength{\parindent}{0pt}
    \setlength{\parskip}{6pt plus 2pt minus 1pt}}
}{% if KOMA class
  \KOMAoptions{parskip=half}}
\makeatother
\usepackage{xcolor}
\IfFileExists{xurl.sty}{\usepackage{xurl}}{} % add URL line breaks if available
\IfFileExists{bookmark.sty}{\usepackage{bookmark}}{\usepackage{hyperref}}
\hypersetup{
  pdftitle={AIS Spoofing},
  hidelinks,
  pdfcreator={LaTeX via pandoc}}
\urlstyle{same} % disable monospaced font for URLs
\setlength{\emergencystretch}{3em} % prevent overfull lines
\providecommand{\tightlist}{%
  \setlength{\itemsep}{0pt}\setlength{\parskip}{0pt}}
\setcounter{secnumdepth}{-\maxdimen} % remove section numbering

\title{AIS Spoofing}
\usepackage{etoolbox}
\makeatletter
\providecommand{\subtitle}[1]{% add subtitle to \maketitle
  \apptocmd{\@title}{\par {\large #1 \par}}{}{}
}
\makeatother
\subtitle{R. Ben Voth, Joey Simone

California State University Maritime Academy

Capstone: Maritime Profession People and Planet

Dr. Bets McNie

8 APR 2024}
\author{}
\date{}

\begin{document}
\maketitle

Approaches to Prevention and Detection of Malicious AIS Data

The modernization of the Maritime Industry has included many lifesaving
improvements to our navigation and communication equipment and
protocols. The \emph{Automatic Identification System (AIS)} has decades
of service facilitating vessel communication and preventing collisions;
the use of AIS Aids to Navigation (AIS AToNS) is very useful for
establishing marks where it is impossible to rig a traditional buoy, or
to temporarily mark a dangerous position while traditional marks are
installed. The AIS, when given true information and interpreted by
competent officers, has saved lives, and will save countless more.
However, the advent of cyberwarfare jeopardizes the safety of mariners
like us, and AIS has minimal security features for preventing attacks. A
solution must be adopted which verifies a message's sender's identity,
that the message has not been tampered with, and that the content of the
message is genuine.

\hypertarget{what-is-ais-spoofing}{%
\section{What is AIS Spoofing?}\label{what-is-ais-spoofing}}

\hypertarget{key-definitions}{%
\subsection{Key definitions}\label{key-definitions}}

\hypertarget{regulations-pertaining-to-ais}{%
\subsection{Regulations pertaining to
AIS}\label{regulations-pertaining-to-ais}}

\hypertarget{signs-of-spoofed-data}{%
\subsection{Signs of spoofed data}\label{signs-of-spoofed-data}}

\hypertarget{who-are-the-major-actors}{%
\section{Who are the Major Actors?}\label{who-are-the-major-actors}}

\hypertarget{china}{%
\subsection{China}\label{china}}

\hypertarget{russia}{%
\subsection{Russia}\label{russia}}

\hypertarget{criminals}{%
\subsection{Criminals}\label{criminals}}

\hypertarget{what-are-the-major-issues}{%
\section{What are the Major Issues?}\label{what-are-the-major-issues}}

\hypertarget{cyber-terrorism}{%
\subsection{Cyber Terrorism}\label{cyber-terrorism}}

Mariners, particularly the watchstanding bridge officers, rely on AIS
data to make safe navigation decisions; false AIS data necessarily
decreases the safety of those navigation decisions. AIS messages are
entirely insecure, so malicious actors can perpetrate attacks with
effects ranging from confusion and delay to grounding and collisions.
Wimpenny Et Al. provide a brief list of possible attacks from spoofed
AIS messages:

\begin{quote}
``• Driving a vessel off course or re-routing it by spoofing the
presence of virtual aids to navigation (VAtoN) via a spoofed `AIS
Message 21'. VAtoN have no physical presence, but instead provide data
via an AIS message which is used to display the presence of AtoN
electronically on an ECDIS. Legitimate uses for VAtoN include the
marking of new hazards (such as a dangerous wreck) whilst authorities
arrange for a physical AtoN, or in instances where a physical AtoN is
not appropriate.

• Spoofing a vessel's reported position by providing it with false DGNSS
corrections via a spoofed `AIS Message 17'. Such an attack could
interfere with a vessel's ability to navigate safely and has the
potential to lead a vessel into harm (depending on how this data is used
on the bridge).

• Performing denial of service attacks by misusing AIS channel
management broadcasts via spoofing `AIS Message 22'. Such broadcasts
could be used to instruct all AIS transceivers within range to cease
operation or switch their AIS broadcasting onto an inappropriate
maritime VHF channel, thereby interfering with other maritime
communications.

The above are all particularly nefarious forms of attack as, unlike
spoofing the presence of a physical AtoN or vessel, these spoofing
attacks cannot be verified visually or against data from another system,
such as radar, and the mariner is reliant solely on AIS data.'' (2022,
p. 334).
\end{quote}

Such attacks are difficult to prevent or detect because the AIS does not
verify senders or content, and the bridge has no way of knowing whether
the received information can be trusted or not. A ship falling for the
attack and being rerouted into a dangerous location is the worst-case
scenario, but even the news or threat that such attacks occur may cause
bridge teams to overreact and to disregard all AIS data in the future.
In either situation, the functional utility of the AIS is diminished,
again preventing its use in aiding safe navigation.

\hypertarget{disguising-illegal-activity}{%
\subsection{Disguising illegal
activity}\label{disguising-illegal-activity}}

As discussed in the subsection on criminals in the Major Actors section,
AIS spoofing can be used to conceal evidence of crime at sea. Criminals
can use false AIS data to hide their ship's location, route, and their
business. One such case is the Russian ``shadow tanker'' fleet, a group
of smugglers who transmit false location data to conceal their route,
their origin, and the transfer of their cargo. In September of 2023,
Alaric Nightingale reported on the tanker \emph{Turba} lightering oil
from her own hold to another tanker, \emph{Simba}, while falsely
reporting that their own position was in a place miles away from her.
``The precise reasons for any individual transfer are never clear, but
the moves add a degree of separation and obfuscation for those
ultimately purchasing the consignments.'' (Nightingale, 2023). This is
just one part of the shell game by which a sanctioned and cash strapped
country like Russia can achieve their strategic goals. Oil can be loaded
secretly onto a vessel transmitting a false name, travel to a country
different from what the AIS says is the case, rendezvous with a clean
vessel, and sell the oil. The more ubiquitous case is \emph{IUUF:
Illegal/Unreported/Unregulated Fishing}. Max Krüger's 2019 presentation
discussed how illegal fishers could use fraudulent data to claim that
they are not fishing vessels, that they aren't engaged in fishing, or to
not appear to be fishing in an unauthorized zone. ``Accordingly, for a
fishing vessel, performing illegal activities creates some incentive to
cover its own identity and/or activity by AIS type spoofing.'' (Krüger,
2019). This, of course, is a financial motivation. To sell illegal
species to rich epicures, to fish a greater quantity than authorized, or
in the waters of a country where certain fish are more populous could be
lucrative enough to risk punishment for being caught.

\hypertarget{impersonation}{%
\subsection{Impersonation}\label{impersonation}}

Impersonation describes falsifying the AIS data of a vessel so that a
vessel may pretend to be another, or that land stations can lie about
the whereabouts of a vessel. One particularly worrying case is the
incident with the HMS Defender in the early weeks of the Russia-Ukraine
war in 2021.

\begin{quote}
\emph{``In the early hours of June 19, the site's tracking data showed
the HMS Defender and a Dutch frigate, HNLMS Evertsen, approaching the
port of Sevastopol in Crimea. The strange thing is, they weren't
there.'' (Bateman, 2021).}
\end{quote}

This constitutes a cyber attack perpetrated presumably by the Russian
Military to create misleading AIS data to be uploaded to the website
\emph{Marine Traffic}, a platform which makes available up to date ship
position info of ships globally. This was part of a misinformation
campaign where ``Russia's Defence Ministry claimed it fired warning
shots and dropped bombs to deter the Royal Navy vessel.'' (Bateman,
2021). The simulated data was used to skew public perception of the
nature of the conflict in the water, and the political ramifications if
the spoofing was not detected could have included a shooting war between
the Russian Federation and the United Kingdom. Although this is the most
extreme and high-profile example to date, this should illustrate the
need to verify that the position information in AIS reports is true, and
that the sender is the vessel.

\hypertarget{obfuscation}{%
\subsection{Obfuscation}\label{obfuscation}}

\hypertarget{what-solutions-have-been-proposed}{%
\section{What solutions have been
proposed?}\label{what-solutions-have-been-proposed}}

\hypertarget{sensor-based-detection-technologies}{%
\subsection{Sensor based detection
technologies}\label{sensor-based-detection-technologies}}

\hypertarget{ai-and-blockchain-solutions}{%
\subsection{AI and Blockchain
solutions}\label{ai-and-blockchain-solutions}}

\hypertarget{public-key-cryptography}{%
\subsection{Public Key Cryptography}\label{public-key-cryptography}}

\hypertarget{drawbacks-of-this-solution}{%
\section{Drawbacks of this Solution}\label{drawbacks-of-this-solution}}

\hypertarget{positive-outcomes-of-this-solution}{%
\section{Positive Outcomes of this
Solution}\label{positive-outcomes-of-this-solution}}

\hypertarget{conclusion}{%
\section{Conclusion}\label{conclusion}}

It is obvious that a solution is necessary to upgrade the AIS to a
modern standard of cybersecurity to secure the safety of global trade
and the environment from those who wish harm on them. For the people in
this room, this means that you have a choice between having stricter
controls on the AIS data you receive and being able to trust the
validity of the ship reports and the ATS AToNs placed by local
authorities, or constantly worrying that cyber-attacks are feeding you
false data. The solution outlined herein can only be successfully
implemented with development from the private sector, regulation from
the public sector, and the welcome of the maritime industry. Invest in
the technology, lobby for legislation, and be willing to learn the new
system when it is developed. The consequences for failing are grim.

References

Androjna, A., Pavić, I., Gucma, L., Vidmar, P., \& Perkovič, M. (2024).
AIS data manipulation in the illicit global oil trade. \emph{Journal of
Marine Science and Engineering}, \emph{12}(1), 6-.
https://doi.org/10.3390/jmse12010006

Androjna, A., \& Perkovič, M. (2021). Impact of Spoofing of Navigation
Systems on Maritime Situational Awareness. \emph{Transactions on
Maritime Science}, \emph{10}(2), Article 2.
https://doi.org/10.7225/toms.v10.n02.w08

Androjna, A., Perkovič, M., Pavic, I., \& Mišković, J. (2021). AIS Data
Vulnerability Indicated by a Spoofing Case-Study. \emph{Applied
Sciences}, \emph{11}(11), Article 11.
https://doi.org/10.3390/app11115015

Bateman, T. (2021, June 28). \emph{HMS Defender: AIS spoofing is opening
up a new front in the war on reality}. Euronews.
https://www.euronews.com/next/2021/06/28/hms-defender-ais-spoofing-is-opening-up-a-new-front-in-the-war-on-reality

Bloomberg. (2023, September 27). \emph{Fake coordinates and tanker
tricks expose shadowy russian oil trade}. gCaptain.
https://gcaptain.com/fake-coordinates-and-tanker-tricks-expose-shadowy-russian-oil-trade/

Boudehenn, C., Cexus, J.-C., Abdelkader, R., Lannuzel, M., Jacq, O.,
Brosset, D., \& Boudraa, A. (2023). Holistic Approach of Integrated
Navigation Equipment for Cybersecurity at Sea. In C. Onwubiko, P.
Rosati, A. Rege, A. Erola, X. Bellekens, H. Hindy, \& M. G. Jaatun
(Eds.), \emph{Proceedings of the International Conference on
Cybersecurity, Situational Awareness and Social Media} (pp. 75--86).
Springer Nature. https://doi.org/10.1007/978-981-19-6414-5\_5

Courtnell, J. (2023, February 10). \emph{What is spoofing? Your complete
guide (+ 4 key AIS spoofing typologies)}. Pole Star.
https://www.polestarglobal.com/resources/what-is-spoofing-your-complete-guide-4-key-ais-spoofing-typologies

d'Afflisio, E., Braca, P., \& Willett, P. (2021). Malicious AIS spoofing
and abnormal stealth deviations: A comprehensive statistical framework
for maritime anomaly detection. \emph{IEEE Transactions on Aerospace and
Electronic Systems}, \emph{57}(4), 2093--2108.
https://doi.org/10.1109/TAES.2021.3083466

Diakun, B., \& Meade, R. (2024, March 8). \emph{Black Sea shipping hit
by rising Russian GPS jamming}.
https://www.lloydslist.com/LL1148493/Black-Sea-shipping-hit-by-rising-Russian-GPS-jamming

Duan, Y., Huang, J., Lei, J., Kong, L., Lv, Y., Lin, Z., Chen, G., \&
Khan, M. K. (2023). AISChain: Blockchain-based AIS data platform with
dynamic bloom filter tree. \emph{IEEE Transactions on Intelligent
Transportation Systems}, \emph{24}(2), 2332--2343.
https://doi.org/10.1109/TITS.2022.3188691

Goudosis, A., \& Katsikas, S. (2020). Secure AIS with identity-based
authentication and encryption. \emph{TransNav, International Journal on
Marine Navigation and Safety Od Sea Transportation}, \emph{14}(2),
287--298. https://doi.org/10.12716/1001.14.02.03

Goudossis, A., \& Katsikas, S. K. (2019). Towards a secure automatic
identification system (AIS). \emph{Journal of Marine Science and
Technology}, \emph{24}(2), 410--423.
https://doi.org/10.1007/s00773-018-0561-3

Guo, S. (2014). \emph{Space-based detection of spoofing AIS signals
using doppler frequency}. \emph{9121}, 912108-912108--6.
https://doi.org/10.1117/12.2050448

Harris, M. (2024, March 7). \emph{How illegal fishing ships hide}.
https://nautil.us/how-illegal-fishing-ships-hide-526064/

Iphar, C., Napoli, A., \& Ray, C. (2015). \emph{Detection of false AIS
messages for the improvement of maritime situational awareness}. 1--7.
https://doi.org/10.23919/OCEANS.2015.7401841

Iphar, C., Ray, C., \& Napoli, A. (2020). Data integrity assessment for
maritime anomaly detection. \emph{Expert Systems with Applications},
\emph{147}, 113219-. https://doi.org/10.1016/j.eswa.2020.113219

Jones, A., Koehler, S., Jerge, M., Graves, M., King, B., Dalrymple, R.,
Freese, C., \& Von Albade, J. (2023). BATMAN: A brain-like approach for
tracking maritime activity and nuance. \emph{Sensors (Basel,
Switzerland)}, \emph{23}(5), 2424-. https://doi.org/10.3390/s23052424

Katsilieris, F., Braca, P., \& Coraluppi, S. (2013). \emph{Detection of
malicious AIS position spoofing by exploiting radar information}.
1196--1203. https://ieeexplore.ieee.org/document/6641132

Kelly, P. (2022). A novel technique to identify AIS transmissions from
vessels which attempt to obscure their position by switching their AIS
transponder from normal transmit power mode to low transmit power mode.
\emph{Expert Systems with Applications}, \emph{202}, 117205-.
https://doi.org/10.1016/j.eswa.2022.117205

Kessler, G. C. (2020). Protected AIS: A demonstration of capability
scheme to provide authentication and message integrity. \emph{TransNav,
International Journal on Marine Navigation and Safety Od Sea
Transportation}, \emph{14}(2), 279--286.
https://doi.org/10.12716/1001.14.02.02

Krüger, M. (2019). Detection of AIS spoofing in fishery scenarios.
\emph{2019 22th International Conference on Information Fusion
(FUSION)}, 1--7. https://doi.org/10.23919/FUSION43075.2019.9011328

Louart, M., Szkolnik, J.-J., Boudraa, A.-O., Le Lann, J.-C., \& Le Roy,
F. (2023). Detection of AIS messages falsifications and spoofing by
checking messages compliance with TDMA protocol. \emph{Digital Signal
Processing}, \emph{136}, 103983-.
https://doi.org/10.1016/j.dsp.2023.103983

Ray, C., Gallen, R., Iphar, C., Napoli, A., \& Bouju, A. (n.d.).
\emph{DeAIS project: Detection of AIS spoofing and resulting risks
\textbar{} ieee conference publication \textbar{} ieee xplore}.
Retrieved March 21, 2024, from
https://ieeexplore.ieee.org/abstract/document/7271729

Sasson, H. (2021, September 13). \emph{What is vessel identity
laundering and why does it matter?} Windward.
https://windward.ai/blog/north-korean-sanctions-evasion-identity-laundering-explained/

Schuler, M. (2024, March 5). \emph{Planet Labs secures major U.S. Navy
contract for advanced maritime monitoring}. gCaptain.
https://gcaptain.com/planet-labs-secures-major-niwc-pacific-contract/

Wimpenny, G., Šafář, J., Grant, A., \& Bransby, M. (2022). Securing the
automatic identification system (AIS): Using public key cryptography to
prevent spoofing whilst retaining backwards compatibility. \emph{The
Journal of Navigation}, \emph{75}(2), 333--345.
https://doi.org/10.1017/S0373463321000837

\end{document}
