\documentclass{article}[12pt,letterpaper]
\usepackage{graphicx} % Required for inserting images
\usepackage{geometry}[margin=1in]
\usepackage{ragged2e}
\usepackage{titling}
\usepackage{setspace}
\usepackage[utf8]{inputenc}
\usepackage{hyperref,csquotes}
\usepackage{xpatch}
%\usepackage{biblatex-mla}
\usepackage[american]{babel}
\usepackage[style=mla,autocite=footnote]{biblatex}
\DeclareAutoCiteCommand{footnote}[f]{\footcite}{\footcites}
\setlength{\bibhang}{1.5em}
\makeatletter
\DeclareCiteCommand{\fullcitefoot}
  {\renewcommand{\finalnamedelim}
   {\ifnum\value{liststop}>2 \finalandcomma\fi\addspace\&\space}%
   \list{}
   {\setlength{\leftmargin}{\bibhang}%
     \setlength{\itemindent}{-\leftmargin}%
     \setlength{\itemsep}{\bibitemsep}%
     \setlength{\parsep}{\bibparsep}}\item}
  {\usedriver
    {\DeclareNameAlias{sortname}{default}}
    {\thefield{entrytype}}\finentry}
  {\item}
  {\endlist\global\undef\bbx@lasthash}

\DeclareCiteCommand{\fullcitebib}
  {\renewcommand{\finalnamedelim}
   {\ifnum\value{liststop}>2 \finalandcomma\fi\addspace\&\space}%
   \list{}
   {\setlength{\leftmargin}{\bibhang}%
     \setlength{\itemindent}{-\leftmargin}%
     \setlength{\itemsep}{\bibitemsep}%
     \setlength{\parsep}{\bibparsep}}\item}
  {\usedriver
    {}
    {\thefield{entrytype}}\finentry}
  {\item}
  {\endlist\global\undef\bbx@lasthash}
\makeatother
\addbibresource{law.bib}
\doublespacing
\title{Significant Sentence}
\author{Joey Simone}

\pagestyle{empty}

\date{Admiralty Law -- 30 January 2024}
\begin{document}

\begin{flushleft}
\thetitle

\thedate

\theauthor
\end{flushleft}

``Courts stressed the importance of avoiding any hard line for what is “adjoining.”'' \fullcitebib{schoenbaum_admiralty_2019} %you have to make a copy of the law.bib in your own directory and change the page number in the database

I am always interested in the edge cases of the laws and protections surrounding the workers, water’s edge to use a turn of phrase. Immediately upon beginning to read about the “Maritime Situs” and “Maritime Status” tests I began to puzzle out the different ways that unscrupulous employers may attempt to prove that a worker injured or killed in certain border areas may not be entitled to the compensation named in the law. It therefore piques my interest to know that the Supreme Court expanded the definition to avoid such anomalies, shutting down my thought exercise of finding a room or border or annex in a port in which by some coincidence of circumstance a longshoreman may lose his protection. No hard line can be made, so any space that dock workers may work is a space that can satisfy the situs test. An inclusive gray that fills in all the gaps and plugs all the loopholes and crannies.
\end{document}
