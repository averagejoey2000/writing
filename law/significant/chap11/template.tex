\documentclass{article}[12pt,letterpaper]
\usepackage{graphicx} % Required for inserting images
\usepackage{geometry}[margin=1in]
\usepackage{ragged2e}
\usepackage{titling}
\usepackage{setspace}
\usepackage[utf8]{inputenc}
\usepackage{hyperref,csquotes}
\usepackage{xpatch}
%\usepackage{biblatex-mla}
\usepackage[american]{babel}
\usepackage[style=mla,autocite=footnote]{biblatex}
\DeclareAutoCiteCommand{footnote}[f]{\footcite}{\footcites}
\setlength{\bibhang}{1.5em}
\makeatletter
\DeclareCiteCommand{\fullcitefoot}
  {\renewcommand{\finalnamedelim}
   {\ifnum\value{liststop}>2 \finalandcomma\fi\addspace\&\space}%
   \list{}
   {\setlength{\leftmargin}{\bibhang}%
     \setlength{\itemindent}{-\leftmargin}%
     \setlength{\itemsep}{\bibitemsep}%
     \setlength{\parsep}{\bibparsep}}\item}
  {\usedriver
    {\DeclareNameAlias{sortname}{default}}
    {\thefield{entrytype}}\finentry}
  {\item}
  {\endlist\global\undef\bbx@lasthash}

\DeclareCiteCommand{\fullcitebib}
  {\renewcommand{\finalnamedelim}
   {\ifnum\value{liststop}>2 \finalandcomma\fi\addspace\&\space}%
   \list{}
   {\setlength{\leftmargin}{\bibhang}%
     \setlength{\itemindent}{-\leftmargin}%
     \setlength{\itemsep}{\bibitemsep}%
     \setlength{\parsep}{\bibparsep}}\item}
  {\usedriver
    {}
    {\thefield{entrytype}}\finentry}
  {\item}
  {\endlist\global\undef\bbx@lasthash}
\makeatother
\addbibresource{law.bib}
\doublespacing
\title{Significant Sentence}
\author{Joey Simone}

\pagestyle{empty}

\date{Admiralty Law -- DD MON YYYY}
\begin{document}

\begin{flushleft}
\thetitle

\thedate

\theauthor
\end{flushleft}

``Because of this, it is customary for vessels to procure hull and P \& I insurance to protect the vessel owner against the consequences of pilot negligence.'' \fullcitebib{schoenbaum_admiralty_2019} %you have to make a copy of the law.bib in your own directory and change the page number in the database

I have always thought of the relationship between the Master and the Pilot and the Ship as to an Attending Physician, Consulting Specialist, and a Patient. If I trust both of these professionals to attend to my care, but the Consultant makes a mistake at a critical moment leading to damage to me, I would want to sue him; but Doctors have tremendous student debt, and even the doctors with the most successful private practices don’t have enough assets to pay actual and punitive damages, so I’d really be suing his insurance company. If a pilot damages a ship, even though pilots make a lot of money, they don’t have buy me a new ship money, and the book explains that pilots would up the price of their services to include their own, for lack of a better word, malpractice insurance premiums. Although I may feel that this would mean that pilots have a financial license to be irresponsible in their ship handling, I recognize that it would be extremely difficult to pursue them financially for their mistakes.
\end{document}
