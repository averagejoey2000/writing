\documentclass{article}[12pt,letterpaper]
\usepackage{graphicx} % Required for inserting images
\usepackage{geometry}[margin=1in]
\usepackage{ragged2e}
\usepackage{titling}
\usepackage{setspace}
\usepackage[utf8]{inputenc}
\usepackage{hyperref,csquotes}
\usepackage{xpatch}
%\usepackage{biblatex-mla}
\usepackage[american]{babel}
\usepackage[style=mla, backend=biber]{biblatex}
\addbibresource{law.bib}
\doublespacing
\title{Significant Sentence}
\author{Joey Simone}
\date{Admiralty Law -- \today}

\pagestyle{empty}
\begin{document}

\begin{flushleft}
\thetitle

\thedate

\theauthor
\end{flushleft}

``The subject of historic shipwrecks in the Area is addressed by UNCLOS Article 149: All objects of an archaeological and historical value found in the Area shall be preserved or disposed of for the benefit of mankind as a whole, particular regard being paid to the preferential rights of the State or country of origin, or the State of cultural origin, or the State of historical and archaeological origin.'' \fullcite{schoenbaum_admiralty_2019}[p. 789]

As an avid lover of sea shanties, \textit{Treasure Island}, and \textit{Assassin's Creed Black Flag}, it is a little disappointing to me that the ancient law of ``finders-keepers'' does not apply to what the UN decides are shipwrecks of historical significance.
Perhaps some distinction can be made that although the wreck of a sunken Spanish Galleon in international waters may be of archaeological value as a historical remnant of colonialism, the native gold in the hold is not, and can therefore be all mine.

\printbibliography

\end{document}
