\documentclass{article}[12pt,letterpaper]
\usepackage{graphicx} % Required for inserting images
\usepackage{geometry}[margin=1in]
\usepackage{ragged2e}
\usepackage{titling}
\usepackage{setspace}
\usepackage[utf8]{inputenc}
\usepackage{hyperref,csquotes}
\usepackage{xpatch}
%\usepackage{biblatex-mla}
\usepackage[american]{babel}
\usepackage[style=mla,autocite=footnote]{biblatex}
\DeclareAutoCiteCommand{footnote}[f]{\footcite}{\footcites}
\setlength{\bibhang}{1.5em}
\makeatletter
\DeclareCiteCommand{\fullcitefoot}
  {\renewcommand{\finalnamedelim}
   {\ifnum\value{liststop}>2 \finalandcomma\fi\addspace\&\space}%
   \list{}
   {\setlength{\leftmargin}{\bibhang}%
     \setlength{\itemindent}{-\leftmargin}%
     \setlength{\itemsep}{\bibitemsep}%
     \setlength{\parsep}{\bibparsep}}\item}
  {\usedriver
    {\DeclareNameAlias{sortname}{default}}
    {\thefield{entrytype}}\finentry}
  {\item}
  {\endlist\global\undef\bbx@lasthash}

\DeclareCiteCommand{\fullcitebib}
  {\renewcommand{\finalnamedelim}
   {\ifnum\value{liststop}>2 \finalandcomma\fi\addspace\&\space}%
   \list{}
   {\setlength{\leftmargin}{\bibhang}%
     \setlength{\itemindent}{-\leftmargin}%
     \setlength{\itemsep}{\bibitemsep}%
     \setlength{\parsep}{\bibparsep}}\item}
  {\usedriver
    {}
    {\thefield{entrytype}}\finentry}
  {\item}
  {\endlist\global\undef\bbx@lasthash}
\makeatother
\addbibresource{law.bib}
\doublespacing
\title{Significant Sentence}
\author{Joey Simone}

\pagestyle{empty}

\date{Admiralty Law -- 21 February 2024}
\begin{document}

\begin{flushleft}
\thetitle

\thedate

\theauthor
\end{flushleft}

``The scope of the doctrine was defined as follows: [A] shipowner, relying on the expertise of another party (the contractor), enters into a contract whereby the contractor agrees to perform services without supervision or control by the shipowner; the improper, unsafe or incompetent execution of such services would foreseeably render the vessel unseaworthy or bring into play a preexisting unseaworthy condition; and the shipowner would thereby be exposed to liability regardless of fault.'' \fullcitebib{schoenbaum_admiralty_2019} %you have to make a copy of the law.bib in your own directory and change the page number in the database

I like looking at the ways that the responsibility of the captain or of the company are moved or shared. This feels similar to the case of giving a bar pilot the conn of the ship, or similar to the way a hospitalist may consult with a specialist doctor over the case of a patient’s care. When a tugboat pulls a barge, the tug crew is responsible for the safe navigation as a barge. When pushing ahead or alongside, the barge, per the COLREGS, is lit in a way resembling a regular ship, where the tug and the barge together are lit as a large ship. A ship, such as the Western Eagle on the previous page, is surrendering control of itself to be the barge of the tug. However, as the doctrine above that I’ve quoted says, the trust does not release the owner from liability. A master retains responsibility for the ship, the attending physician is always liable for the patient’s health.
\end{document}
