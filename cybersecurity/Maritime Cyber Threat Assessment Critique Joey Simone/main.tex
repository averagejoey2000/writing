\documentclass{article}[letterpaper,12pt]
\usepackage{graphicx} % Required for inserting images
\usepackage{geometry}[margin=1in]
\usepackage{setspace}
%\doublespacing


\title{Maritime Cyber Threat Assessment Critique Joey Simone}
\author{Joey Simone}
\date{\today}

\begin{document}
%\twocolumn
\maketitle
Responding to Skyler Williams and Giovanni Cerrito
\begin{itemize}
    \item In defining who possible attackers are, they don't give any guesses as to what group might want to cause damage, or why. Specify which people and what damage.
    \item With regards to those seeking money, they don't differentiate stealing money from any particular asset account or revenue stream. It's different to try to intercept cashnet payments as opposed to stealing credit card information at the bistro or trying to impersonate service companies. Specify what money is being taken.
    \item Define what makes Duo hard to hack, what makes that more difficult than another 2FA platform. \begin{itemize}
        \item The next line mentions the danger of phishing, but Duo 2FA makes successful logins with stolen credentials more difficult, as victims must be fooled twice. First fool them into giving up their login information, then fool them into authorizing a duo push.
        \item Are Cal Maritime cadets and professors more easily or less easily tricked than students and profs at other schools? Cubicle workers and office managers at other companies?
    \end{itemize}
    \item In the vision of what would happen if a student or professor's credentials are so compromised, the group makes some tenuosly supported claims of what an attacker could do once inside. To claim that they could install ransomware in a meaningful way onto the school's backend, (which I must assume is what is meant by "damage the whole system") would require that students and faculty have the permissions to do so with their user accounts.
    \item Explain why a buffer overflow is a reasonable attack to forecast.
    \item They give the personally identifying information of students and parents and credit card information as a vulnerability, not a target. Just having stuff in the bank doesn't prevent the vault door from closing correctly.
    \item What do they mean by shut down the IT department's ability to stop an attack?
    \item I agree that phishing emails are a likely method an attacker might choose because of how low the risk and time commitment is for making and sending phishing emails. I disagree that it's highly likely that a phishing email is highly likely to cause a catastrophic failure.
\end{itemize}
\end{document}
